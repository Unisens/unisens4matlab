\documentclass{unisens}

\usepackage[latin1]{inputenc}
\usepackage[T1]{fontenc}
\usepackage[ngerman]{babel}
\usepackage{tabularx, varioref, mit_bib}


\newcommand{\xmlattribute}[1]{\texttt{#1}}
\setcounter{secnumdepth}{0}

\title{Unisens und Matlab}
\subject{Installationsanleitung}
\author{Malte Kirst}

\begin{document}
\maketitle

	\section{Voraussetzungen}

Folgende Punkte sind Voraussetzung f�r die Integration von Unisens 2.0 in Matlab:
\begin{itemize}
	\item Matlab 7.1 oder h�her ist installiert
	\item Java JRE 1.5.0 oder h�her wird verwendet
	\item org.unisens.jar ist vorhanden
	\item org.unisens.ri.jar ist vorhanden
\end{itemize}

Die Dateien org.unisens.jar (Interface) und org.unisens.ri.jar (Referenzimplementierung) und einige Beispieldaten
finden sich  im Paket \texttt{unisens4matlab} im Downloadbereich auf der Unisens-Webseite \url{http://www.unisens.org}.


	\section{Installation}

		\subsection{Java}

Die verwendete Matlab-Version muss das Java JRE 1.5.0 oder h�her verwenden. �berpr�fbar ist dies im Pfad \code{\$matlabroot\textbackslash sys\textbackslash java\textbackslash jre\textbackslash win32\textbackslash } 
oder durch den Befehl \code{version -java} im Matlab-Command-Window.

Sollte eine niedrigere Java-Version installiert sein, kann Unisens nicht verwendet werden. Steht keine Matlab-Version zur Verf�gung, die Java JRE 1.5.0 oder h�her verwendet, kann das vorhandene Matlab auch mit einer anderen JRE gestartet werden. Daf�r muss die JRE 1.5.0 oder h�her von \url{http://www.java.com/de/download/} herunter geladen und installiert werden. Anschlie�end kann Matlab mit der neuen JRE gestartet werden (Matlab-Dokumentation). Es kann jedoch vorkommen, dass einzelne Funktionen innerhalb von Matlab mit einer anderen JRE nicht funktionieren.


		\subsection{Matlab}

Die JAR-Dateien \texttt{org.unisens.jar} und \texttt{org.unisens.ri.jar} m�ssen in Matlab eingebunden werden. Dieses kann statisch oder dynamisch erfolgen. Um die JAR-Dateien statisch (dauerhaft) hinzuzuf�gen, muss die Datei \texttt{classpath.txt} mit dem Befehl \code{edit classpath.txt} im Command-Window ge�ffnet werden. Anschlie�endem wird der  vollst�ndige Pfad zu den beiden JAR-Datei am Ende der Datei eingef�gt und die Datei gespeichert. Sollte das Speichern fehlschlagen, sind die Dateirechte zu �berpr�fen. Ab dem n�chsten Neustart von Matlab sind diese �nderungen dauerhaft gespeichert. 

Alternativ k�nnen beide JAR-Dateien dynamisch durch Eingabe von \code{javaaddpath(\textsl{PFAD})} im Command-Window eingebunden werden, wobei \code{\textsl{PFAD}} der vollst�ndige Pfad zur jeweiligen JAR-Datei ist. Bei einem Matlab-Neustart gehen dynamische �nderungen verloren.

Wenn der Befahl \code{javaclasspath} die JAR-Dateien auflistet, war das Einbinden erfolgreich.



	\section{Test}

Um den Test durchf�hren zu k�nnen, m�ssen sich die M-Dateien aus diesem Paket im aktuellen Verzeichnis oder in einem von Matlab durchsuchten Verzeichnis befinden. Der Befehl \code{publish unisens\_example} generiert ein ausf�hrliches Beispiel. Um Informationen zu einer Unisens-Datei anzuzeigen, wird der Befehl \code{unisens\_get\_entry\_info} verwenden. Weitere Testdateien befinden sich auf der Unisens-Webseite \url{http://www.unisens.org}.

\code{unisens\_get\_data} liest Daten aus einem SignalEntry, \code{unisens\_plot} plottet die ersten 20 Sekunden aller SignalEntries in eine Grafik. 


	\section{Dokumentation}

Die Matlab-Unisens-Funktionen sind alle dokumentiert, mit dem Kommando \code{help \textsl{BEFEHL}} wird die Hilfe angezeigt. Alle Java-Methoden k�nnen auch direkt von Matlab aus aufgerufen werden. Die Dokumentation der Java-API befindet sich auf \url{http://www.unisens.org}.

\end{document}
